\documentclass{article}

\usepackage[margin=1.0in]{geometry}
\usepackage{graphicx}

\title{Electrical and Computer Engineering \\ 
Professional Practice Co-op Report
}
\author{
\textbf{General Electric Aviation Systems} \\
\\
\textit{Author}:     Brian Rieder\\
\textit{Advisor}:    Cindy Quillen\\
\textit{Supervisor}: Seth McDonald
}
\date{17 June 2016}

\begin{document}
% title page
\maketitle
\begin{center}
  \includegraphics[width=5cm,height=5cm,keepaspectratio]{General_Electric_1930.png}
\end{center}
\pagenumbering{gobble}
\newpage

% table of contents
\tableofcontents
\pagenumbering{gobble}
\newpage

% document body
\pagenumbering{arabic}

% -------------------------------------- COMPANY DESCRIPTION
\section{Company Description}
\subsection{Company History}
\subsubsection{General Electric}
General Electric is a multinational conglomerate corporation founded in 1892 in Schenectady, New York by Thomas Edison. The headquarters is still located in Schenectady with Jeffrey Immelt residing as the CEO and Chairman. 
As of 2015, there are a multitude of subsets of the company located across the United States that include Appliances, Power and Water, Oil and Gas, Energy Management, Aviation, Healthcare, Transportation, and Capital -- all of which function in the home appliances, financial services, medical equipment, life sciences, pharmaceutical, automotive, software development, and engineering industries.
\subsubsection{GE Aviation}
GE Aviation, one of the aforementioned subsidiaries of General Electric, is headquartered in Evendale, Ohio. GE Aviation is the primary aircraft engine supplier in the country, but also works within the electrical side of the aviation industry on the design and implementation of products such as Flight Management Systems (FMS) and Integrated Modular Architectures (IMA) in its avionics facility in Grand Rapids, Michigan. 
GE Aviation maintains contracts with international government and private customers such as the US Navy, Boeing, COMAC, and Dassault Systèmes. 
GE Aviation Systems, specifically, maintains a joint venture with COMAC, a Chinese aerospace company, called Aviage and worked together to create the C919 passenger jet.
\subsection{Assignment Placement Description}
While working for General Electric, I was placed in GE Aviation Systems, formerly Smiths Aerospace, in Grand Rapids, Michigan.
However, in an interesting turn of events this rotation, I was placed in a department that did not specialize in avionics, programming, or any sort of electronics, for that matter.
This specific assignment saw me being placed within the newly founded Engineering Program and Product Quality department - specifically within the data and analytics subsection. Our primary goal was to identify pain points within the company and to pinpoint the changes necessary within processes to reduce cost on a multitude of programs at sites across the United States.
Because of the corporate nature of this rotation and my position so close to an executive, I was actually in a position to communicate with twelve different Aviation Systems (most recognizably those associated with avionics) sites and their site leaders in order to facilitate change. As such, I spent much of my time not working where I was located, but rather on the phone and communicating with people from across the country.

% -------------------------------------- WORK DESCRIPTION
\section{Work Description}
\subsection{Data Analytics Engineer}
My primary assignment on this rotation was actually in a role that we deemed to be called an Analytics Research and Solutions Engineer. The main purpose of my role was to identify issues within the current data schema and to develop solutions to the issue that ranged from directly addressing processes within the company to manually compiling reports or scripts that recified the data.
Because of my specialization in the sanitization of data over this rotation, I developed an expertise in evaluating data sets and developing more efficient manners of verifying data. My primary goal was to generate, validate, and provide data sets that were able to indicate First Time Yield, Nonconformance rates, Scrap and Rework quantifiability, and the rate of Out of Box Failure occurrences within a dashboard developed within the visualization tool called Spotfire. When data proved to be inadequate or problematic within the tool, it was my job to verify that data, identify the root cause, and to right the wrongs that appeared within it.
\subsection{PostgreSQL Developer}
Because my interests were heavily software based through all of my rotations, I was afforded the opportunity to do a significant amount of development on the back-end of our tool in order to compile data for reports. Most notably, I spent a significant amount of time automating the generation of a cost report by part number through a SQL view that aggregated nine different data sources.
While it was unfortunate that I didn't get to do any actual development as I have in previous rotations, I saw the value in being given the opportunity to work with these tools and to develop my understanding of database structures and development as a whole.

% -------------------------------------- WORK RESPONSIBILITIES
\section{Work Responsibilities}
While I was given a significant amount of responsibility within this rotation, I would actually argue that it was less than my previous rotation. Looking back on my rotation within my internal R\&D team, I was given the power to own specific aspects of an aircraft model as well as the ability to push changes into a production environment of a tool that a multitude of people used.
This rotation I was given the power to develop my own tools, but was met by a system of checks, balances, and necessary verification within the team. While this is a necessary evil - I am not nearly as experienced as a number of other people within the team - it gave me a layer of security that made me feel not nearly as responsible for what I was doing. 
There was constant deferral to teammates, but this also meant constant interaction and teamwork within the team.
The only exception to this was when I was given the opportunity to lead investigation of Out of Box failures within the company and the nature in which people recorded data within the company. To do so, I called a number of different site leaders and discussed differences in process, tool usage, and potential future adjustments - an entire dialogue of which I was required to interact with someone in a higher position than me, ask them all the right questions, and to summarize a report which was used to develop the dashboard that will be used in the fourth quarter of 2016 to display Out of Box failure statistics.
All-in-all, the pressure and responsibility placed on me this rotation makes my first assignment pale in comparison -- however, compared to my previous rotation, it was an entirely different ballgame. I spent a significant amount of time developing myself professionally in my last rotation, but this time I felt that I just rode on the coattails of my manager - not an inherently bad thing, but certainly less trying on my professional abilities.


% -------------------------------------- USE OF PRIOR COURSEWORK
\section{Use of Prior Coursework}
A bit of unfortunate circumstance put me in a position in which I wasn't applying any electrical or computer engineering experience. Most of my time was spent within Excel and on the phone within this rotation - primarily employing my previous experience on rotation with professionalism and team communication.
The sole part of this rotation that could arguably be relevant to my degree was the time that I spent within PostgreSQL and VBA development. Because of familiarity with program flow and general programming practices, I was able to pick up development within these specific environments very quickly. If I was able to spend more time on these specific tasks, I believe the rotation would have been more enjoyable - but I was forced to spend the time that I could, glean off the relevant experience, and make the best of the rotation without applying previous coursework.

% stupid formatting...
\newpage
% -------------------------------------- NEW KNOWLEDGE
\section{New Knowledge and Skills Acquired}
\subsection{Business Processes and Metric Analytics}
Since my role was primarily within analytics of company data, I learned a significant amount about the processes internal to the company - specifically the tracking and calculation of nonconformances, First Time Yield, and out of box failures. Through understanding the concept of nonconformances - concisely, any defect within a process or part that causes straying from a predefined process - I was able to perform analysis on company data to determine trends, identify problem parts, and quantify issues caused by specific problems within processes by cost. Furthermore, I was able to analyze and rework our calculation of First Time Yield, the percentage of parts that can pass through the entire production process without nonconforming or failure of any inspections, by analyzing how the company recorded non-serialized parts versus serialized parts and proposing a solution that summed conforming parts over a total number while accounting for overlap. All-in-all, I was able to develop a much more in-depth understanding of company processes and also weigh in on quantification of data - overall, further developing business prowess and understanding the functionality of business internal processes.

\subsection{Programming Expertise}
As opposed to my previous rotation, programming was not my primary takeaway. However, I was afforded some opportunity to look into development:
\subsubsection{PostgreSQL}
All of my previous experience in database programming came from my previous rotation working within MSSQL. Using PostgreSQL within this team gave our visualization dashboard a method of storing, querying, and visualizing all of the data that was constructed from the base tables from various sources around the company. Because of the multiple source issues, much of my work went into aggregating data by forming it into a common format and compiling larger views that were used within our visualization. PostgreSQL proved to be more similar to my personal experience within MySQL than that of MSSQL, but I found that there was a significant issue with obsolescence due to the tools just being outdated.
\subsubsection{Visual Basic for Applications}
Along with database work, I spent an enormous amount of time working within Excel and formatting tables to be loaded into the data lake. In order to do so, there were several occassions in which I had to write scripts in VBA in order to operate on data. Since I'd not programmed in VBA previously, I found that it synergized well with Excel, but also found that it was one of the most abhorrent programming languages I'd ever worked in. I value the experience to work with a tool that isn't inherently what I like so I can develop an understanding of its functionality, but in this specific instance I found that the quirks of the language were more of a productivity killer than an improvement on how other languages operate. I gleaned some useful programming experience from its usage, but I'd rather avoid it when possible in the future - despite the fact that it's everywhere Excel is present.

% -------------------------------------- AWARDS AND HONORS
\section{Awards and Honors}
\begin{description}
  \item[Above and Beyond Bronze Award (100 USD)] \hfill \\
    Award to express gratitude for efforts exceeding the expectation or requested detail in an assignment. This was awarded for efforts in my previous rotation in which I was able to pick up various technologies related to a home-grown tool that enhanced efficiency when developing models of new-age aircraft. The specific tool is detailed in my previous rotation's report, but was awarded retroactively due to the timing of my previous rotation's end.
  \item[Above and Beyond Silver Award (100 USD)] \hfill \\
    Award to express gratitude for efforts exceeding the expectation or requested detail in an assignment. This was awarded for efforts in developing a view within our database that could generate a compiled cost report instantaneously and maintain in real time to replace the four hour process of compiling a report without 100\% certainty that the report was correct.
\end{description}

\end{document}
