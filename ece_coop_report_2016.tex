\documentclass{article}

\usepackage[margin=1.0in]{geometry}
\usepackage{graphicx}

\title{Electrical and Computer Engineering \\ 
Professional Practice Co-op Report
}
\author{
\textbf{General Electric Aviation Systems} \\
\\
\textit{Author}:     Brian Rieder\\
\textit{Advisor}:    Cindy Quillen\\
\textit{Supervisor}: Seth McDonald
}
\date{17 June 2015}

\begin{document}
% title page
\maketitle
\begin{center}
  \includegraphics[width=5cm,height=5cm,keepaspectratio]{General_Electric_1930.png}
\end{center}
\pagenumbering{gobble}
\newpage

% table of contents
\tableofcontents
\pagenumbering{gobble}
\newpage

% document body
\pagenumbering{arabic}

% -------------------------------------- COMPANY DESCRIPTION
\section{Company Description}
\subsection{Company History}
\subsubsection{General Electric}
General Electric is a multinational conglomerate corporation founded in 1892 in Schenectady, New York by Thomas Edison. The headquarters is still located in Schenectady with Jeffrey Immelt residing as the CEO and Chairman. 
As of 2015, there are a multitude of subsets of the company located across the United States that include Appliances, Power and Water, Oil and Gas, Energy Management, Aviation, Healthcare, Transportation, and Capital -- all of which function in the home appliances, financial services, medical equipment, life sciences, pharmaceutical, automotive, software development, and engineering industries.
\subsubsection{GE Aviation}
GE Aviation, one of the aforementioned subsidiaries of General Electric, is headquartered in Evendale, Ohio. GE Aviation is the primary aircraft engine supplier in the country, but also works within the electrical side of the aviation industry on the design and implementation of products such as Flight Management Systems (FMS) and Integrated Modular Architectures (IMA) in its avionics facility in Grand Rapids, Michigan. 
GE Aviation maintains contracts with international government and private customers such as the US Navy, Boeing, COMAC, and Dassault Systèmes. 
GE Aviation Systems, specifically, maintains a joint venture with COMAC, a Chinese aerospace company, called Aviage and worked together to create the C919 passenger jet.
\subsection{Assignment Placement Description}
While working for General Electric, I was placed in GE Aviation Systems, formerly Smiths Aerospace, in Grand Rapids, Michigan.
However, in an interesting turn of events this rotation, I was placed in a department that did not specialize in avionics, programming, or any sort of electronics, for that matter.
This specific assignment saw me being placed within the newly founded Engineering Program and Product Quality department - specifically within the data and analytics subsection. Our primary goal was to identify pain points within the company and to pinpoint the changes necessary within processes to reduce cost on a multitude of programs at sites across the United States.
Because of the corporate nature of this rotation and my position so close to an executive, I was actually in a position to communicate with twelve different Aviation Systems (most recognizably those associated with avionics) sites and their site leaders in order to facilitate change. As such, I spent much of my time not working where I was located, but rather on the phone and communicating with people from across the country.

% -------------------------------------- WORK DESCRIPTION
\section{Work Description}
\subsection{Data Analytics Engineer}
My primary assignment on this rotation was actually in a role that we deemed to be called an Analytics Research and Solutions Engineer. The main purpose of my role was to identify issues within the current data schema and to develop solutions to the issue that ranged from directly addressing processes within the company to manually compiling reports or scripts that recified the data.
Because of my specialization in the sanitization of data over this rotation, I developed an expertise in evaluating data sets and developing more efficient manners of verifying data. My primary goal was to generate, validate, and provide data sets that were able to indicate First Time Yield, Nonconformance rates, Scrap and Rework quantifiability, and the rate of Out of Box Failure occurrences within a dashboard developed within the visualization tool called Spotfire. When data proved to be inadequate or problematic within the tool, it was my job to verify that data, identify the root cause, and to right the wrongs that appeared within it.
\subsection{PostgreSQL Developer}
Because my interests were heavily software based through all of my rotations, I was afforded the opportunity to do a significant amount of development on the back-end of our tool in order to compile data for reports. Most notably, I spent a significant amount of time automating the generation of a cost report by part number through a SQL view that aggregated nine different data sources.
While it was unfortunate that I didn't get to do any actual development as I have in previous rotations, I saw the value in being given the opportunity to work with these tools and to develop my understanding of database structures and development as a whole.

% -------------------------------------- WORK RESPONSIBILITIES
\section{Work Responsibilities}
While I was given a significant amount of responsibility within this rotation, I would actually argue that it was less than my previous rotation. Looking back on my rotation within my internal R&D team, I was given the power to own specific aspects of an aircraft model as well as the ability to push changes into a production environment of a tool that a multitude of people used.
This rotation I was given the power to develop my own tools, but was met by a system of checks, balances, and necessary verification within the team. While this is a necessary evil - I am not nearly as experienced as a number of other people within the team - it gave me a layer of security that made me feel not nearly as responsible for what I was doing. 
There was constant deferral to teammates, but this also meant constant interaction and teamwork within the team.
The only exception to this was when I was given the opportunity to lead investigation of Out of Box failures within the company and the nature in which people recorded data within the company. To do so, I called a number of different site leaders and discussed differences in process, tool usage, and potential future adjustments - an entire dialogue of which I was required to interact with someone in a higher position than me, ask them all the right questions, and to summarize a report which was used to develop the dashboard that will be used in the fourth quarter of 2016 to display Out of Box failure statistics.
All-in-all, the pressure and responsibility placed on me this rotation makes my first assignment pale in comparison -- however, compared to my previous rotation, it was an entirely different ballgame. I spent a significant amount of time developing myself professionally in my last rotation, but this time I felt that I just rode on the coattails of my manager - not an inherently bad thing, but certainly less trying on my professional abilities.


% -----LEFT OFF HERE!!!!!

% -------------------------------------- USE OF PRIOR COURSEWORK
\section{Use of Prior Coursework}
Unfortunately, I found that I was only able to apply one class directly to this rotation, but rather found that the majority of the expertise I was able to apply was from my previous rotation as well as my personal programming pursuits. 
Due to my familiarity with the functionality of the FMS within an aircraft from my time working on the P-8 Poseidon, I found that I was able to apply some of the performance and connectivity conventions that I'd seen previously to the model that I was working on for the generic business jet. The P-8, though, is a Boeing 737 based military aircraft, so there was a significant size difference between the two that led to the necessity of certain components that wouldn't necessarily be present otherwise. The primarily application of my classes, though, came with the programming of the web application that I was working on.
ECE 264, Advanced C Programming, turned out to be valuable exposure that spurred my launch into programming pursuits in my spare time to expose me to HTML, CSS, PHP, and JavaScript that I used during the entirety of my time programming this session. Not only that, but I was able to practice valuable debugging techniques in that class that allowed me to be more efficient and effective during my rotation.
Some minor electrical exposure helped me with some of the model analysis, but I found that I would have been able to use Google to acquire sufficient knowledge to do the calculations that were necessary for this rotation.

% stupid formatting...
\newpage
% -------------------------------------- NEW KNOWLEDGE
\section{New Knowledge and Skills Acquired}
\subsection{Aircraft Subsystems}
Modeling efforts gave me massive exposure to the inner workings of an aircraft defined by ATA 100 chapters. As I was assigned ATA chapters 26, 28, and 38 -- Fire Protection, Fuel, and Water and Waste, respectively -- that is where the majority of the information that I learned fell into place. Not only do I have a strong understanding of all the hardware and software associated with each chapter, I also have an understanding of how those subsystems fall within the larger system view of the aircraft and how they interact with other chapters dispersed across the aircraft.
The knowledge I gained through the modeling efforts is going to be immensely helpful provided that I pursue an occupation in aerospace and even more so if I pursue an occupation in avionics and IMA. The largest downside of this exposure, though, is that none of the knowledge gained on this team -- outside of interpersonal interaction and working on a team -- will never be applicable outside of the aerospace industry, thus forcing me into somewhat of a niche if I want to fully utilize everything I learned in this rotation.

\subsection{Programming Expertise}
Programming was my primary takeaway this rotation, as I was given an enormous amount of exposure to various languages.
\subsubsection{PostgreSQL}
I had very little direct interaction with JavaScript coming into this rotation, but by the end I considered myself a fully functional developer in the language. JavaScript is a high-level interpreted language that is executed primarily on the client-side in web based applications. Not only did I learn the applications and basis of the language, but I also found that overuse of JavaScript in the wrong application can cause massive security holes. Because of this, I was constantly on the lookout for computations or authorizations that should be incorporated on the server-side and reducing the reliance on the client's browser to do much of the computation -- giving me a strong bearing on security applications that are associated with web application development.
\subsubsection{Visual Basic for Applications}
Before this rotation, I had never worked with PHP in my life. PHP is an extremely powerful and dynamic programming language that is executed on the server-side of a web application -- I also found it to be one of the easiest programming languages that I have ever learned. I found that PHP is remarkably resilient and tolerant of user errors, but that also results in unexpected behavior at times where it isn't desired. I grew to love PHP and how powerful it was, but that was primarily due to my ability to directly interact with all of the stored data for the aircraft model from the language.
\subsubsection{MSSQL}
The underlying data skeleton of our web application, MSSQL gave us a method of storing, querying, and editing all of the data that constructed the models that we worked with on a daily basis. MSSQL is a Microsoft variant of SQL, the structured querying language. We were able to construct a database comprised of tables and craft queries that allowed us to get exactly the data we were looking for in any specific application. I had never worked so closely with databases before this rotation, so this experience was immensely valuable due to the growing need for handling large sets of data in an efficient manner.

% -------------------------------------- AWARDS AND HONORS
\section{Awards and Honors}
\begin{description}
  \item[Above and Beyond Bronze Award (100 USD)] \hfill \\
    Award to express gratitude for efforts exceeding the expectation or requested detail in an assignment. This was awarded for efforts in my previous rotation in which I was able to pick up various technologies related to a home-grown tool that enhanced efficiency when developing models of new-age aircraft. The specific tool is detailed in my previous rotation's report, but was awarded retroactively due to the timing of my previous rotation's end.
  \item[Above and Beyond Silver Award (100 USD)] \hfill \\
    Award to express gratitude for efforts exceeding the expectation or requested detail in an assignment. This was awarded for efforts in developing a view within our database that could generate a compiled cost report instantaneously and maintain in real time to replace the four hour process of compiling a report without 100\% certainty that the report was correct.
\end{description}

\end{document}
